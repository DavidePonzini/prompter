\section{Sfide e Potenzialità dell'Adozione dell'IA Generativa}
    \subsection{Saper identificare le sfide nell'adozione dell'IA Generativa}
    \subsection{I pregiudizi e informazioni false presenti nei dati di addestramento di un LLM}
    \subsection{Controllo sugli output generati e prevenire derive imprevedibili o comportamenti ``allucinati''}
    \subsection{Generazione automatica di contenuti, plagio, copyright e trasparenza}
    \subsection{Rischi di sicurezza e abuso: come prevenire l'uso improprio o dannoso dell'IA Generativa}
    \subsection{Impatto sulle dinamiche sociali e ambientali nella vita quotidiana e all'interno delle realtà aziendali}
    \subsection{Identificare le potenzialità nell'adozione dell'IA Generativa}
    \subsection{Automazione della creazione di contenuti}
    \subsection{Potenziamento delle capacità creative degli utenti attraverso strumenti di co-creazione e generazione di idee innovative}
    \subsection{Alta personalizzazione e adattività di contenuti in tempo reale in funzione del contesto e delle esigenze specifiche}
    \subsection{Miglioramento dell'accessibilità e dell'inclusione facilitando l'accesso a informazioni, servizi e contenuti}
    \subsection{Supporto alle attività tramite l’utilizzo di assistenti virtuali}
