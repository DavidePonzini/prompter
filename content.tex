\section{Introduzione}
    \subsection{Competenze del Prompter}
    \subsection{Sfide e opportunita dell’IA conversazionale}
        \subsubsection{Il ruolo del prompter e del prompting nella società}
    \subsection{Sovranità digitale}
    \subsection{Conclusione}
    
\section{Intelligenza artificiale e Chatbot}
    \subsection{Intelligenza Artificiale (IA)}
    \subsection{IA simbolica, IA sub-simbolica e IA neurale}
    \subsection{I chatbot}
    \subsection{Interagire con un chatbot}

\section{Large Language Models e IA Generativa}
    \subsection{Large Language Models (LLM) e le loro principali caratteristiche}
    \subsection{IA Generativa}
    \subsection{Strumenti e piattaforme per il Prompt Engineering}
    \subsection{Strumenti di IA Generativa disponibili oggi sul mercato}
        \subsubsection{Applicazioni pratiche}
    \subsection{Concetti base dell’architettura Transformer}
    \subsection{LLM basati su architettura Transformer}
        \subsubsection{LLM, architettura Transformer e machine learning}
        \subsubsection{Caratteristiche rilevanti dei LLM}
        \subsubsection{Come funzionano i LLM}
    \subsection{Come avviene il processo di interpretazione dei prompt}
    \subsection{Il concetto di ``embedding''}
    \subsection{``Punti di attenzione'' e ``fine-tuning''}
    \subsection{I principali parametri di configurazione di un LLM}
        \subsubsection{Temperatura}
        \subsubsection{Top p (nucleus sampling)}
        \subsubsection{Lunghezza massima}
        \subsubsection{Sequenze di stop}
        \subsubsection{Penalità di frequenza}
        \subsubsection{Penalità di presenza}
    \subsection{Esempio pratico di uso con GPT}
    \subsection{Esempio di interazione con le API GPT-3 in Python}
    \subsection{Personalizzare attraverso piattaforme web}
    \subsection{Personalizzare Chat-GPT}
    \subsection{Principi e tecniche di prompt engineering}
        \subsubsection{Come funzionano i prompt}
        \subsubsection{Interpretazione dei prompt}
        \subsubsection{Embedding}
        \subsubsection{Apprendimento non supervisionato e supervisionato}
    \subsection{Rappresentazione interna}
    \subsection{Rappresentazione del testo}
        \subsubsection{Punti di attenzione}
        \subsubsection{Fine tuning}
    \subsection{Risposta ai prompt}
        \subsubsection{Esempi}
    \subsection{Prompt impliciti e prompt espliciti}
        \subsubsection{Prompt impliciti}
        \subsubsection{Esempio di prompt implicito}
        \subsubsection{Prompt espliciti}
        \subsubsection{Esempio di prompt esplicito}
    \subsection{Prompt espliciti vs. promt impliciti: esempi}
    \subsection{Il fenomeno delle ``allucinazioni''}
    \subsection{Strumenti di IA Generativa attraverso API}

\section{Formulazione dei prompt}
    \subsection{Introduzione al prompt engineering}
    \subsection{Prompt semplici ed efficaci}
    \subsection{Prompt impliciti ed espliciti}
    \subsection{Precisione e chiarezza nella formulazione dei prompt}
        \subsubsection{Riduzione dell’ambiguità}
        \subsubsection{Guida specifica per il modello}
        \subsubsection{Miglioramento dell’interazione uomo-macchina}
        \subsubsection{Ottimizzazione del training e del tuning del modello}
        \subsubsection{Esempi}
    \subsection{Zero-shot, one-shot, few-shot}
        \subsubsection{Zero-shot learning}
        \subsubsection{One-shot learning}
        \subsubsection{Few-shot learning}
    \subsection{Prompt a catena di pensieri}
        \subsubsection{Esempi}
        \subsubsection{Vantaggi del prompt a catena di pensieri}
    \subsection{Prompting Costituzionale}
        \subsubsection{Sfide del prompting costituzionale}
        \subsubsection{La lingua del prompt}
    \subsection{Attività quotidiane di scrittura, suggerimenti creativi}
    \subsection{Zero-shot, one-shot e few-shot learning}
    \subsection{Prompt efficaci per generare contenuti marketing, reportistica e materiali formativi, oppure per la scrittura di relazioni, test, ricerche}
    \subsection{Prompt a catena di pensieri}
    \subsection{Prompting costituzionale}
    \subsection{L'impatto della lingua utilizzata per formulare i prompt }

\section{Ottimizzazione dei prompt}
    \subsection{Strategie per affinare i prompt}
    \subsection{Tecniche di split testing e analisi iterativa dei prompt}
    \subsection{Tecniche di utilizzo di prompt parametrizzati}
    \subsection{Sperimentare con diversi prompt e a monitorare i risultati}

\section{Utilizzo dell'IA Generativa nella vita quotidiana}
    \subsection{Saper utilizzare assistenti virtuali e chatbot per fornire supporto}
    \subsection{Saper utilizzare IA Generativa per creare contenuti di base}
    \subsection{Saper utilizzare IA Generativa per progetti personali e per attività quotidiane}
    \subsection{IA Generativa per migliorare l’accessibilità e l’inclusione}
    \subsection{Utilizzare l’IA Generativa in modo etico e sicuro}
    \subsection{Prompt specifici per la creazione di contenuti complessi e per analisi dei dati e simulazioni}
    \subsection{Migliorare l’accessibilità, creare contenuti multimediali, simulazioni, realtà virtuale}
    \subsection{Co-creazione di progetti creativi complessi, supporto alla progettazione e alla programmazione}
    \subsection{Supporto alla soluzione di problemi e nei processi decisionali}
    \subsection{Supporto alla estrazione di conoscenza da grandi quantità di dati}
    \subsection{Valutare e scegliere le migliori soluzioni da adottare per la propria realtà aziendale}

\section{Etica, sicurezza, sostenibilità e normative sull'IA}
    \subsection{Privacy e sicurezza online}
    \subsection{Valutare l’informazione online}
    \subsection{Problematiche etiche e i rischi nell'uso dell’IA Generativa}
    \subsection{Impatto ambientale e sostenibilità dell’IA Generativa}
    \subsection{La normativa europea ``AI Act'' e le normative vigenti sull’IA}

\section{Sfide e Potenzialità dell'Adozione dell'IA Generativa}
    \subsection{Saper identificare le sfide nell'adozione dell'IA Generativa}
    \subsection{I pregiudizi e informazioni false presenti nei dati di addestramento di un LLM}
    \subsection{Controllo sugli output generati e prevenire derive imprevedibili o comportamenti ``allucinati''}
    \subsection{Generazione automatica di contenuti, plagio, copyright e trasparenza}
    \subsection{Rischi di sicurezza e abuso: come prevenire l'uso improprio o dannoso dell'IA Generativa}
    \subsection{Impatto sulle dinamiche sociali e ambientali nella vita quotidiana e all'interno delle realtà aziendali}
    \subsection{Identificare le potenzialità nell'adozione dell'IA Generativa}
    \subsection{Automazione della creazione di contenuti}
    \subsection{Potenziamento delle capacità creative degli utenti attraverso strumenti di co-creazione e generazione di idee innovative}
    \subsection{Alta personalizzazione e adattività di contenuti in tempo reale in funzione del contesto e delle esigenze specifiche}
    \subsection{Miglioramento dell'accessibilità e dell'inclusione facilitando l'accesso a informazioni, servizi e contenuti}
    \subsection{Supporto alle attività tramite l’utilizzo di assistenti virtuali}

\section{Leadership nell'adozione dell'IA Generativa per il management}
    \subsection{Formare colleghi di studio e lavoro sugli usi avanzati e casi d'uso complessi dell'IA Generativa}
    \subsection{Promuovere la sperimentazione e l'innovazione aziendale}
    \subsection{Saper creare un set di esercizi interattivi per i colleghi che operano in reparti aziendali differenti}
    \subsection{Saper guidare la creazione di policy e linee guida etiche per un uso appropriato}
    \subsection{Saper usare in modo efficace l’IA Generativa per la comunicazione interna ed esterna delle organizzazioni}
    \subsection{Utilizzare in modo efficace l’IA Generativa per il management delle risorse e all’ottimizzazione dei processi}