\section{Ottimizzazione dei prompt}
    Questo capitolo analizza strategie per ottimizzare i prompt e migliorare le interazioni con modelli di linguaggio come ChatGPT o Claude. Vengono esplorate tecniche come split testing, analisi iterativa dei prompt, utilizzo di prompt parametrizzati, test di usabilità con utenti, analisi delle metriche di coinvolgimento e prestazioni, e implementazione di un feedback loop continuo. Vengono forniti esempi e suggerimenti pratici per utenti individuali. Inoltre, il capitolo affronta il problema delle ``allucinazioni'', ovvero quando il modello genera informazioni non accurate o inventate. Vengono discusse le possibili cause, come limitazioni del modello, bias nei dati di addestramento e ottimizzazione per la probabilità del linguaggio. Sono proposte misure per ridurre le allucinazioni, tra cui fine-tuning mirato, controllo dei prompt, validazione umana e tecnologie di mitigazione.

    \subsection{Strategie per affinare i prompt}
        Per ottimizzare i prompt e migliorare le interazioni con i modelli di linguaggio, è essenziale implementare strategie che permettano non solo di perfezionare i prompt iniziali, ma anche di valutare e testare l'efficacia di tali prompt nel generare risposte utili e precise. Possibili strategie sono:
        \begin{itemize}
            \item \textbf{Split testing}: lo split testing, o A/B testing, è una tecnica per confrontare due o più versioni di un prompt per determinare quale produce risultati migliori. Questo metodo può essere utilizzato per testare variazioni nel linguaggio, nella struttura o nel dettaglio del prompt.
            
            \item \textbf{Analisi Iterativa dei prompt}: questa tecnica implica il continuo miglioramento dei prompt attraverso iterazioni successive basate sul feedback e sui risultati ottenuti. Dopo ogni iterazione, il prompt viene modificato per affrontare le debolezze identificate.
            
            \item \textbf{Utilizzo di prompt parametrizzati}: i prompt parametrizzati permettono di inserire variabili dinamiche nel prompt stesso, che possono essere adattate in base al contesto specifico dell'interazione. Questo rende i prompt più flessibili e potenzialmente più efficaci.
            
            \item \textbf{Testing di usabilità con utenti}: il testing di usabilità implica il coinvolgimento diretto degli utenti finali per testare e valutare l'efficacia dei prompt in condizioni reali di utilizzo.
            
            \item \textbf{Analisi delle metriche di coinvolgimento e prestazioni}: misurare sistematicamente come i prompt influenzano il coinvolgimento degli utenti e le prestazioni del modello, analizzando metriche come il tempo di risposta, la frequenza di interazione, e la soddisfazione espressa dagli utenti.
            
            \item \textbf{Feedback loop}: implementare un sistema in cui il feedback degli utenti viene costantemente raccolto e utilizzato per affinare i prompt.
        \end{itemize}
        
        Nel seguito prendiamo in esame ognuna di queste strategie con i relativi metodi specifici per testare e affinare i prompt.
    
    \subsection{Tecniche di split testing e analisi iterativa dei prompt}
        % TODO
    
    \subsection{Tecniche di utilizzo di prompt parametrizzati}
        % TODO
    
    \subsection{Sperimentare con diversi prompt e a monitorare i risultati}
        % TODO
