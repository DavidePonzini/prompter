\section{Introduzione}
    Il \textit{prompting} rappresenta una metodologia emergente dell'Intelligenza Artificiale Generativa (IA-Gen), di fondamentale importanza per massimizzare l'efficacia dei ``Modelli linguistici di grandi dimensioni'' (Large Language Models), cioè sistemi progettati per elaborare e generare linguaggio naturale, dove il termine “grandi dimensioni” indica l'estensione e la complessità di tali modelli, che sono addestrati su grandi quantità di testo proveniente da molteplici fonti, che includono libri, articoli, siti web, programmi e altri contenuti scritti, permettendo loro di acquisire una comprensione approfondita e sfumata del linguaggio, dimostrando una capacità significativa nel generare testi creativi e realistici che possono essere difficilmente distinguibili da quelli scritti da esseri umani. Tali testi possono essere inoltre adattati e ottimizzati per specifici domini o applicazioni, tramite tecniche come il fine-tuning, rendendoli molto versatili per una vasta gamma di utilizzi.

    Il prompting consiste nella formulazione di \textit{prompt} -- input testuali che innescano una risposta del modello di IA -- strutturati in modo da massimizzare la qualità e la pertinenza delle risposte generate. Questa pratica implica l'uso di tecniche specifiche per affinare e adattare i prompt, migliorando così l'efficacia dei modelli di linguaggio in varie applicazioni.

    Per mettere in evidenza gli aspetti metodologici che andremo a discutere per dialogare in modo efficace con sistemi di IA-Gen, parleremo nel seguito anche di ``Metodologie di conversazione con l'Intelligenza Artificiale'' o, più semplicemente, di \textit{IA Conversazionale (IA-Conv)}. In un contesto di flussi informativi sempre più vasti e pervasivi, queste metodologie emergono infatti come strumento cruciale per orientare e affinare le potenzialità creative e decisionali dell'IA, rendendola uno strumento sempre più versatile e accurato. Ed è in questo contesto che si afferma la figura del \textit{prompter}, colei o colui che non necessariamente ha competenze tecniche approfondite sui meccanismi di funzionamento dell'IA, ma deve avere buone capacità comunicative e creative per tradurre requisiti complessi in prompt (input testuali) efficaci per guidare i modelli di IA-Gen verso obiettivi specifici, ottenendo risposte pertinenti e valide al problema posto. L'importanza dell'IA-Conv e del ruolo del prompter non riguarda quindi solo l'aspetto tecnico, ma anche l'uso responsabile e il progresso etico dell'IA in settori cruciali della società.

    I prompter possono essere cittadini digitali, imprenditori, studenti, umanisti digitali, docenti, tecnici, ricercatori, che possono apportare prospettive diverse e integrare valori culturali e umani nello svolgimento delle attività quotidiane, lavorative ma non solo. Alcuni esempi di attività che possono svolgere sono:
    \begin{itemize}
        \item \textbf{Cittadini digitali}: Possono utilizzare prompt semplici ed espliciti per facilitare ricerche, richiedere assistenza o informazioni di vario tipo all'IA nel contesto della vita quotidiana. Ad esempio, immaginate una persona che utilizza l'IA per trovare rapidamente informazioni su servizi locali, pianificare viaggi o gestire la propria agenda. Grazie ai prompt ben strutturati, l'IA può rispondere in modo efficace e preciso, migliorando la qualità della vita quotidiana.

        \item \textbf{Imprenditori}: Possono impiegare prompt più complessi e mirati per applicazioni aziendali come l'analisi di dati, la generazione di contenuti, la risoluzione di problemi o lo sviluppo di nuove idee commerciali. Ad esempio, un imprenditore può utilizzare l'IA per analizzare tendenze di mercato, sviluppare strategie di marketing personalizzate o automatizzare compiti ripetitivi, liberando tempo per attività strategiche.

        \item \textbf{Studenti}: Potrebbero usare prompt guidati per approfondire argomenti di studio, ottenere spiegazioni, generare contenuti come saggi o relazioni sfruttando le conoscenze dell'IA. L'IA può diventare quindi un tutor virtuale, offrendo supporto personalizzato e immediato, migliorando l'apprendimento e stimolando la curiosità intellettuale.

        \item \textbf{Umanisti digitali}: Possono formulare prompt creativi per esplorare applicazioni dell'IA in ambiti artistici, letterari o culturali. Ad esempio, l'IA può essere utilizzata per generare storie, poesie o altri contenuti creativi, collaborando con artisti e scrittori per creare opere uniche e innovative.

        \item \textbf{Docenti, tecnici e ricercatori}: Potrebbero sfruttare prompt avanzati per finalità didattiche o di ricerca, come risolvere problemi tecnici, problemi di codifica, analizzare dati scientifici o sviluppare nuovi algoritmi. L'IA può accelerare il processo di scoperta scientifica, offrendo nuovi strumenti per l'analisi dei dati e la modellazione teorica.
    \end{itemize}

    Questi sono solo alcuni esempi della figura del prompter e di come un prompter potrebbe utilizzare le metodologie e le tecniche di conversazione con l'IA per creare soluzioni di IA accurate ed efficienti,  considerando aspetti come la mitigazione delle discriminazioni derivanti dai dati di addestramento dei sistemi di IA-Gen, cruciali per garantire che le risposte generate siano giuste e inclusive, garantendo inoltre il rispetto delle diverse sensibilità culturali.

    Il prompter deve essere inoltre in grado di tradurre concetti, valori e principi derivanti dalle discipline umanistiche (filosofia, etica, studi culturali, etc.) in requisiti e linee guida che possano essere effettivamente implementati nei sistemi di IA. Questo include la promozione della responsabilità e della sostenibilità a lungo termine dell'IA, assicurando che la tecnologia sia utilizzata in modo etico e consapevole.

    Il prompter rappresenta così un elemento chiave di un paradigma innovativo, in cui competenze tecniche e consapevolezza multidisciplinare si uniscono per guidare l'evoluzione dell'IA, massimizzandone i benefici per l'intera società e valorizzandone le opportunità di progresso sociale, culturale, etico ed inclusivo. Proviamo a delineare nel seguito quali dovrebbero essere le competenze di questa figura emergente.
    
    \subsection{Competenze del Prompter}
        Il prompter non è solamente un tecnico; è anche un comunicatore e un etico. Deve comprendere non solo come funzionano i modelli di IA a livello algoritmico (questa conoscenza può variare in funzione della tipologia del prompter e dell'uso che questo fa dell'IA-Conv), ma anche come i loro output influenzano e interagiscono con il mondo reale. Questo richiede una sensibilità verso le implicazioni etiche delle tecnologie AI, specialmente in termini di bias, privacy e impatti sociali. Il prompter deve quindi essere dotato di un acuto senso etico e di una profonda comprensione dei contesti culturali in cui la tecnologia sarà utilizzata.
    
        \begin{itemize}
          \item \textbf{Competenze interdisciplinari}: Il prompter deve possedere una vasta gamma di conoscenze interdisciplinari che spaziano dalla tecnologia all'etica, dalla psicologia alla sociologia. Questo background gli permette di valutare le implicazioni dei prompt usati e di prevedere come le risposte del modello possono essere percepite da diversi gruppi di utenti. Questa competenza è essenziale per prevenire situazioni in cui la tecnologia potrebbe involontariamente perpetuare stereotipi o discriminazioni.
          \item \textbf{Capacità di sintesi e analisi}: I prompter devono essere capaci di sintetizzare complessi requisiti di un problema e aspettative degli utenti in prompt chiari e concisi. Devono anche analizzare i risultati generati dai modelli di IA per assicurarsi che siano corretti e appropriati, adattando i prompt in base alle necessità. Questa abilità è cruciale per garantire che l'IA-Gen fornisca risposte utili e accurate.
          \item \textbf{Abilità comunicative}: Essendo a cavallo tra la tecnologia di IA e le sue applicazioni pratiche, i prompter devono essere eccellenti comunicatori. Questo include la capacità di spiegare concetti tecnici a stakeholder non tecnici, facilitando così una migliore comprensione e collaborazione tra diverse parti interessate. Inoltre, devono essere in grado di tradurre feedback e richieste degli utenti in prompt efficaci.
          \item \textbf{Sensibilità culturale}: In un mondo globalizzato, i prompter devono essere sensibili agli aspetti di inclusione e alle varie norme culturali e sociali. Questo aiuta a garantire che i modelli di IA non solo rispettino le diversità culturali ma siano anche personalizzati per adattarsi efficacemente a contesti specifici. La capacità di comprendere e rispettare le differenze culturali è fondamentale per il successo dell'IA a livello globale.
        \end{itemize}
        In appendice viene riportato il syllabus di competenze del prompter.
 
    \subsection{Sfide e opportunità dell'IA conversazionale}
        L'IA-Conv presenta numerose sfide e opportunità. Tra le sfide principali vi è la necessità di sviluppare prompt che siano non solo tecnicamente validi ma anche eticamente e culturalmente sensibili. Inoltre, i prompter devono essere in grado di adattarsi rapidamente alle evoluzioni tecnologiche e alle nuove capacità dei modelli di IA-Gen.
    
        Le opportunità, d'altra parte, sono immense. L'IA-Conv può rivoluzionare il modo in cui interagiamo con tali sistemi, rendendo la tecnologia più accessibile e utile per una vasta gamma di utenti. Può inoltre contribuire a sviluppare applicazioni innovative che migliorano la qualità della vita, stimolano la creatività e promuovono il progresso sociale.
        
        Guardando al futuro, l'IA-Conv continuerà a evolversi insieme ai progressi di questa disciplina. L'adozione crescente dell'IA in vari settori della società richiederà prompter sempre più competenti e consapevoli delle implicazioni etiche e sociali delle loro scelte. La formazione continua e l'aggiornamento delle competenze saranno essenziali per rimanere al passo con le nuove tecnologie e le loro applicazioni.
        
        Inoltre, l'IA-Conv giocherà un ruolo chiave nella promozione di un'IA etica e responsabile. I prompter saranno in prima linea nell'assicurare che l'IA sia utilizzata in modi che rispettano i diritti umani, promuovano l'inclusione e contribuiscano al bene comune.
    
        \subsubsection{Il ruolo del prompter e del prompting nella società}
            Guardando al presente, l'IA-Conv offre, già da ora, grandi opportunità. Vediamo nel seguito alcuni possibili esempi di utilizzo dell'IA-Conv sia nel pubblico che nel privato.
    
            \begin{itemize}
                \item \textbf{Supporto tecnico amministrativo}: Nei settori pubblici, i prompter possono aiutare a redigere documenti legali o amministrativi utilizzando l'IA per generare bozze iniziali, riducendo il tempo di redazione da giorni a ore. Ad esempio, un funzionario può utilizzare l'IA per preparare report dettagliati basati su dati grezzi raccolti da vari dipartimenti, migliorando l'efficienza e l'accuratezza del lavoro amministrativo. Inoltre, possono essere utilizzati per creare documenti standardizzati come contratti e regolamenti, garantendo conformità e riducendo errori umani.
                
                \item \textbf{Ricerca accademica}: I ricercatori possono utilizzare prompt specifici per estrarre informazioni da vasti database accademici e sintetizzare la letteratura su un argomento specifico, accelerando la revisione della letteratura da settimane a giorni. Per esempio, un ricercatore può impiegare l'IA per trovare articoli scientifici rilevanti, generare riassunti e identificare gap nella ricerca esistente. Inoltre, l'IA può aiutare nella redazione di proposte di ricerca, fornendo suggerimenti su metodologie e riferimenti bibliografici.
                
                \item \textbf{Sviluppo di prodotto}: Nel settore privato, i team di sviluppo prodotto possono utilizzare prompt per guidare l'IA nella generazione di idee innovative basate su tendenze di mercato, migliorando il processo creativo. Ad esempio, un team può utilizzare l'IA per analizzare feedback dei clienti e proporre miglioramenti a prodotti esistenti o sviluppare nuovi prodotti che rispondano meglio alle esigenze del mercato. L'IA può anche simulare scenari di mercato per prevedere il successo di nuovi lanci di prodotti.
                
                \item \textbf{Marketing e pubblicità}: Le aziende possono impiegare i prompter per creare campagne pubblicitarie personalizzate, generando contenuti che risuonino con specifici segmenti di pubblico, riducendo il tempo di creazione e aumentando l'efficacia delle campagne. Ad esempio, un'azienda può utilizzare l'IA per analizzare i dati demografici dei clienti e creare messaggi pubblicitari su misura che catturano meglio l'attenzione dei consumatori. Inoltre, possono essere sviluppate strategie di social media marketing basate su analisi di sentimenti e trend online.
                
                \item \textbf{Analisi dei dati}: I prompter possono guidare l'IA nell'analisi di grandi  insiemi di dati per identificare tendenze e insight, ad esempio analizzando i dati di vendita per suggerire strategie di mercato, accelerando il processo decisionale. Per esempio, un analista può utilizzare l'IA per monitorare le performance di vendita in tempo reale, identificare prodotti più venduti e prevedere le future esigenze di inventario. L'IA può anche supportare nella segmentazione del mercato, permettendo di creare strategie più mirate ed efficaci.
            \end{itemize}
        
    \subsection{Sovranità digitale}
        Un altro aspetto potrebbe essere legato alla sovranità digitale, alla tutela della privacy e alla sicurezza; queste sfide richiedono soluzioni locali che andranno a costituire una parte essenziale dell'infrastruttura di base di uno stato, di una regione o di un comune:
    
        \begin{itemize}
            \item \textbf{Gestione dei beni pubblici}: Le amministrazioni possono utilizzare l'IA per monitorare e gestire infrastrutture e servizi pubblici. Ad esempio, un'IA può analizzare i dati relativi all'uso delle risorse pubbliche, suggerendo ottimizzazioni e manutenzioni preventive.
    
            \item \textbf{Gestione delle risorse ambientali}: L'IA può essere utilizzata per monitorare l'ambiente, analizzare i dati climatici e suggerire politiche di sostenibilità. Ad esempio, un'IA può aiutare a prevedere disastri naturali e ottimizzare la gestione delle risorse idriche.
            
            \item \textbf{Gestione dei beni culturali}: L'IA può supportare nella catalogazione, conservazione e promozione dei beni culturali. Ad esempio, può analizzare i dati sui flussi turistici per ottimizzare la gestione dei musei e dei siti storici.
            
            \item \textbf{Produzione e gestione dell'energia}: Le aziende energetiche possono utilizzare l'IA per ottimizzare la produzione e la distribuzione dell'energia, migliorando l'efficienza e riducendo gli sprechi. Ad esempio, un'IA può analizzare i dati di consumo energetico e suggerire soluzioni per ridurre il fabbisogno energetico.
            
            \item \textbf{Produzione agricola}: Gli agricoltori possono utilizzare l'IA per monitorare le colture, prevedere le condizioni climatiche e ottimizzare l'uso delle risorse. Ad esempio, un'IA può analizzare i dati sul suolo e sul clima per suggerire i momenti migliori per la semina e la raccolta.
            
            \item \textbf{Promozione e gestione del turismo e delle risorse del territorio}: Le amministrazioni locali possono utilizzare l'IA per promuovere il turismo e gestire le risorse del territorio in modo sostenibile. Ad esempio, l'IA può analizzare i dati sui flussi turistici e suggerire strategie per migliorare l'offerta turistica.
            
            \item \textbf{Promozione dell'integrazione e gestione delle Smart City}: Le amministrazioni cittadine possono utilizzare l'IA per promuovere l'integrazione sociale e gestire le infrastrutture in modo intelligente. Ad esempio, l'IA può analizzare i dati sul traffico e sulla mobilità urbana per ottimizzare i trasporti pubblici e ridurre la congestione.
        \end{itemize}
        \subsubsection{Necessità di Disporre di Tecnologie AI in Locale}
            In un mondo sempre più interconnesso, è fondamentale che paesi, aziende e individui dispongano delle tecnologie AI anche in locale, per diversi motivi:
            \begin{itemize}
                \item \textbf{Sicurezza dei Dati}: Mantenere i dati sensibili all'interno del proprio paese o azienda riduce i rischi associati alla loro trasmissione e conservazione su server esterni, proteggendo la privacy e la sicurezza delle informazioni.
    
                \item \textbf{Sovranità Digitale}: Disporre di tecnologie AI in locale garantisce l'indipendenza tecnologica e riduce la dipendenza da fornitori esterni, promuovendo lo sviluppo di competenze locali e l'autonomia strategica.
    
                \item \textbf{Personalizzazione e Controllo}: Avere accesso diretto alle tecnologie AI permette una maggiore personalizzazione delle soluzioni e un controllo più preciso sulle loro implementazioni, adattandole meglio alle specifiche esigenze locali.
    
                \item \textbf{Resilienza Operativa}: In caso di disservizi o problemi di connettività, disporre delle tecnologie AI in locale assicura la continuità operativa e riduce l'impatto di eventuali interruzioni nei servizi.
            \end{itemize}
        
    \subsection{Conclusione}
        Il prompt engineering e la figura del prompter stanno rivoluzionando l'interazione con l'IA, rendendola uno strumento versatile e potente nei settori pubblico e privato. Prompter efficaci possono apportare un valore significativo, riducendo i tempi e aumentando la produttività attraverso una migliore interazione con i modelli di IA, integrandosi con tecnologie IoT, supportando i processi decisionali, migliorando il clima aziendale, integrando con software esistenti, automatizzando processi e fornendo supporto multilingue.
    
        Questo volume ci guiderà attraverso le competenze e le tecniche necessarie per diventare un prompter efficace, esplorando esempi concreti in vari settori. Vediamo insieme come il prompt engineering può trasformare la nostra interazione con l'IA e aprirci nuove opportunità nel mondo digitale.
