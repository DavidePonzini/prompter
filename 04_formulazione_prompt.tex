\section{Formulazione dei prompt}
    \subsection{Introduzione al prompt engineering}
    \subsection{Prompt semplici ed efficaci}
    \subsection{Prompt impliciti ed espliciti}
    \subsection{Precisione e chiarezza nella formulazione dei prompt}
        \subsubsection{Riduzione dell’ambiguità}
        \subsubsection{Guida specifica per il modello}
        \subsubsection{Miglioramento dell’interazione uomo-macchina}
        \subsubsection{Ottimizzazione del training e del tuning del modello}
        \subsubsection{Esempi}
    \subsection{Tecniche di prompting}
        \subsubsection{Zero-shot Prompting}
        \subsubsection{Few-shot Prompting}
        \subsubsection{Chain-of-Thought prompting}
        \subsubsection{Meta Prompting}
        \subsubsection{Generate Knowledge Prompting}
        \subsubsection{Prompt Chaining}
        \subsubsection{Retrieval Augmented Generation (RAG)}
        \subsubsection{Directional Stimulus Prompting}
    \subsection{Prompting Costituzionale}
        \subsubsection{Sfide del prompting costituzionale}
        \subsubsection{La lingua del prompt}
    \subsection{Attività quotidiane di scrittura, suggerimenti creativi}
    \subsection{Prompt efficaci per generare contenuti marketing, reportistica e materiali formativi, oppure per la scrittura di relazioni, test, ricerche}
    \subsection{Prompt a catena di pensieri}
    \subsection{Prompting costituzionale}
    \subsection{L'impatto della lingua utilizzata per formulare i prompt }
